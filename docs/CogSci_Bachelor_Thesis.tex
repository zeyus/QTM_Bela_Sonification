\documentclass[10pt,a4paper,onecolumn]{article}
\usepackage{marginnote}
\usepackage{graphicx}
\usepackage{xcolor}
\usepackage{authblk,etoolbox}
\usepackage{titlesec}
\usepackage{calc}
\usepackage{tikz}
\usepackage{hyperref}
\hypersetup{colorlinks,
            urlcolor=[rgb]{0.0, 0.5, 1.0},
            linkcolor=[rgb]{0.0, 0.5, 1.0}}
\usepackage{caption}
\usepackage{tcolorbox}
\usepackage{amssymb,amsmath}
\usepackage{ifxetex,ifluatex}
\usepackage{seqsplit}
% \usepackage{fixltx2e} % provides \textsubscript
\usepackage[backend=biber,style=apa]{biblatex}

\addbibresource{master.bib}
\addbibresource{packages.bib}

% --- Page layout -------------------------------------------------------------
\usepackage[top=3.5cm, bottom=3cm, right=1.5cm, left=1.0cm,
            headheight=2.2cm, reversemp, includemp, marginparwidth=4.5cm]{geometry}

% --- Default font ------------------------------------------------------------
% \renewcommand\familydefault{\sfdefault}

% --- Style -------------------------------------------------------------------
\renewcommand{\bibfont}{\small \sffamily}
\renewcommand{\captionfont}{\small\sffamily}
\renewcommand{\captionlabelfont}{\bfseries}

% --- Section/SubSection/SubSubSection ----------------------------------------
\titleformat{\section}
  {\normalfont\sffamily\Large\bfseries}
  {}{0pt}{}
\titleformat{\subsection}
  {\normalfont\sffamily\large\bfseries}
  {}{0pt}{}
\titleformat{\subsubsection}
  {\normalfont\sffamily\bfseries}
  {}{0pt}{}
\titleformat*{\paragraph}
  {\sffamily\normalsize}


% --- Header / Footer ---------------------------------------------------------
\usepackage{fancyhdr}
\pagestyle{fancy}
\fancyhf{}
%\renewcommand{\headrulewidth}{0.50pt}
\renewcommand{\headrulewidth}{0pt}

\fancyhead[L]{}
\fancyhead[C]{}
\fancyhead[R]{}
\renewcommand{\footrulewidth}{0.25pt}

\fancyfoot[L]{\footnotesize{\sffamily Backström, L. and Ring, L., (2022). Realtime Sonification of Motion Capture Data., .}}


\fancyfoot[R]{\sffamily \thepage}
\makeatletter
\let\ps@plain\ps@fancy
\fancyheadoffset[L]{4.5cm}
\fancyfootoffset[L]{4.5cm}

% --- Macros ---------

\definecolor{linky}{rgb}{0.0, 0.5, 1.0}

\newtcolorbox{repobox}
   {colback=red, colframe=red!75!black,
     boxrule=0.5pt, arc=2pt, left=6pt, right=6pt, top=3pt, bottom=3pt}

\newcommand{\ExternalLink}{%
   \tikz[x=1.2ex, y=1.2ex, baseline=-0.05ex]{%
       \begin{scope}[x=1ex, y=1ex]
           \clip (-0.1,-0.1)
               --++ (-0, 1.2)
               --++ (0.6, 0)
               --++ (0, -0.6)
               --++ (0.6, 0)
               --++ (0, -1);
           \path[draw,
               line width = 0.5,
               rounded corners=0.5]
               (0,0) rectangle (1,1);
       \end{scope}
       \path[draw, line width = 0.5] (0.5, 0.5)
           -- (1, 1);
       \path[draw, line width = 0.5] (0.6, 1)
           -- (1, 1) -- (1, 0.6);
       }
   }

% --- Title / Authors ---------------------------------------------------------
% patch \maketitle so that it doesn't center
\patchcmd{\@maketitle}{center}{flushleft}{}{}
\patchcmd{\@maketitle}{center}{flushleft}{}{}
% patch \maketitle so that the font size for the title is normal
\patchcmd{\@maketitle}{\LARGE}{\LARGE\sffamily}{}{}
% patch the patch by authblk so that the author block is flush left
\def\maketitle{{%
  \renewenvironment{tabular}[2][]
    {\begin{flushleft}}
    {\end{flushleft}}
  \AB@maketitle}}
\makeatletter
\renewcommand\AB@affilsepx{ \protect\Affilfont}
%\renewcommand\AB@affilnote[1]{{\bfseries #1}\hspace{2pt}}
\renewcommand\AB@affilnote[1]{{\bfseries #1}\hspace{3pt}}
\makeatother
\renewcommand\Authfont{\sffamily\bfseries}
\renewcommand\Affilfont{\sffamily\small\mdseries}
\setlength{\affilsep}{1em}


\ifnum 0\ifxetex 1\fi\ifluatex 1\fi=0 % if pdftex
  \usepackage[T1]{fontenc}
  \usepackage[utf8]{inputenc}

\else % if luatex or xelatex
  \ifxetex
    \usepackage{mathspec}
  \else
    \usepackage{fontspec}
  \fi
  \defaultfontfeatures{Ligatures=TeX,Scale=MatchLowercase}

\fi
% use upquote if available, for straight quotes in verbatim environments
\IfFileExists{upquote.sty}{\usepackage{upquote}}{}
% use microtype if available
\IfFileExists{microtype.sty}{%
\usepackage{microtype}
\UseMicrotypeSet[protrusion]{basicmath} % disable protrusion for tt fonts
}{}

\usepackage{hyperref}
\hypersetup{unicode=true,
            pdftitle={Realtime Sonification of Motion Capture Data},
            pdfkeywords={Sonification; Motion Capture; Realtime; Processing},
            pdfborder={0 0 0},
            }
\urlstyle{same}  % don't use monospace font for urls
\usepackage[normalem]{ulem}
\IfFileExists{parskip.sty}{%
\usepackage{parskip}
}{% else
\setlength{\parindent}{0pt}
\setlength{\parskip}{6pt plus 2pt minus 1pt}
}
\setlength{\emergencystretch}{3em}  % prevent overfull lines
\setcounter{secnumdepth}{5}
% Redefines (sub)paragraphs to behave more like sections
\ifx\paragraph\undefined\else
\let\oldparagraph\paragraph
\renewcommand{\paragraph}[1]{\oldparagraph{#1}\mbox{}}
\fi
\ifx\subparagraph\undefined\else
\let\oldsubparagraph\subparagraph
\renewcommand{\subparagraph}[1]{\oldsubparagraph{#1}\mbox{}}
\fi


% tightlist command for lists without linebreak
\providecommand{\tightlist}{%
  \setlength{\itemsep}{0pt}\setlength{\parskip}{0pt}}

% From pandoc table feature
\usepackage{longtable,booktabs,array}
\usepackage{calc} % for calculating minipage widths
% Correct order of tables after \paragraph or \subparagraph
\usepackage{etoolbox}
\makeatletter
\patchcmd\longtable{\par}{\if@noskipsec\mbox{}\fi\par}{}{}
\makeatother
% Allow footnotes in longtable head/foot
\IfFileExists{footnotehyper.sty}{\usepackage{footnotehyper}}{\usepackage{footnote}}
\makesavenoteenv{longtable}



\title{Realtime Sonification of Motion Capture Data}

        \author[1]{Linus Backström}
          \author[1]{Luke Ring}
    
      \affil[1]{Aarhus University}
  \date{\vspace{-5ex}}

\begin{document}
\maketitle

\marginpar{
  \sffamily\small

  \vspace{2mm}

  {\bfseries Software}
  \begin{itemize}
    \setlength\itemsep{0em}
    \item \href{https://github.com/zeyus/QTM\_Bela\_Sonification}{\color{linky}{Repository}} \ExternalLink
  \end{itemize}

  \vspace{2mm}

  {\bfseries Submitted:} October 31, 2022

  \vspace{2mm}
  {\bfseries License}\\
  Authors of papers retain copyright and release the work under a MIT Licence (\href{https://github.com/zeyus/QTM\_Bela\_Sonification/blob/main/LICENSE.md}{\color{linky}{MIT}}).
}

{
\setcounter{tocdepth}{3}
\tableofcontents
}
\hypertarget{abstract}{%
\section{Abstract}\label{abstract}}

adf

\hypertarget{realtime-sonification-of-motion-capture-data}{%
\section{Realtime Sonification of Motion Capture Data}\label{realtime-sonification-of-motion-capture-data}}

Main intro goes here (why what how?)

Single dollars (\$) are required for inline mathematics e.g.~\(f(x) = e^{\pi/x}\)

Double dollars make self-standing equations:

\[\Theta(x) = \left\{\begin{array}{l}
0\textrm{ if } x < 0\cr
1\textrm{ else}
\end{array}\right.\]

For a quick reference, the following citation commands can be used:
- \textcite{zammSynchronizingMIDIWireless2019} -\textgreater{} ``Author et al.~(2001)''
- \autocite{walkerMUSICALSOUNDSCAPESACCESSIBLE2007} -\textgreater{} ``(Author et al., 2001)''
- \autocite{vinkenAuditoryCodingHuman2013,umekSuitabilityStrainGage2017} -\textgreater{} ``(Author1 et al., 2001; Author2 et al., 2002)''

\hypertarget{literature-review-backgroundetc}{%
\section{Literature review (background/etc)}\label{literature-review-backgroundetc}}

Interpersonal coordination describes situations where two or more individuals strive to match or complement each other's actions,
and it is rooted in both the perception and anticipation of motor actions between those individuals \autocite{gerdschmitzSoundJoinedActions2017}.
While perception, anticipation and action have traditionally been viewed as separate phenomena, current theories state that they are all intrinsically connected.
A large part of embodiment research builds on the key hypothesis that merely observing an action triggers internal models of
movement that actively engage the motor system \autocite{gerdschmitzSoundJoinedActions2017}.
As a result, the motor system appears to not only assist with the planning and controlling of one's own actions,
but also serves the secondary function of perceiving and anticipating other people's actions \autocite{gerdschmitzSoundJoinedActions2017}.
This dual-purpose function of the motor system helps with achieving interpersonal coordination and can also result in spontaneous coordination,
where the actions of others may unwittingly affect one's own actions, even after being instructed to ignore them \autocite{demosRockingBeatEffects2012}.

Sonification is defined as the use of nonspeech audio to convey information.
More specifically, sonification is the transformation of data relations into perceived relations in an acoustic
signal for the purposes of facilitating communication or interpretation \autocite{kramerSonificationReportStatus1999}.
While concepts around sonification and audification were not formalized until around the year 1992 when the first
International Conference on Auditory Display (ICAD) was held \autocite{dubusSonificationPhysicalQuantities2011}, practical examples of sonification can
be found throughout history. Water clocks in ancient Greece and medieval China were sometimes constructed to
produce sounds and thereby provide auditory information about the passage of time \autocite{dubusSonificationPhysicalQuantities2011}.
The stethoscope, which is used for listening to sounds made by the heart and lungs as well as other internal
sounds of the body, was invented in 1816 by French physician and amateur musician Rene Laënnec \autocite{roguinReneTheophileHyacinthe2006}.
The Geiger counter developed in 1928 provides perhaps the most characteristic example of sonification through
its function of sonifying levels of radiation. The device detects ionizing radiation and translates it into audible clicks,
where a faster tempo signifies a higher level of radiation. \textcite{dubusSonificationPhysicalQuantities2011} describe the value of the Geiger
counter as ``transposing a physical quantity which is essentially non-visual and pictured in everyone's imagination
as very important because life-threatening, to the auditory modality through clicks with a varying pulse''.

\ldots.

\sout{RQ: To what extent does real-time sonification of object movement effect joint action synchronization between subjects?}

\sout{H\_0: Real-time sonification of object movement has no effect on joint action synchronization between subjects.}

\sout{H\_1: Real-time sonification of object movement effects joint action synchronization between subjects.}

now RQ is about orientation of sonification scheme (task vs sync)

\hypertarget{methods}{%
\section{Methods}\label{methods}}

Describe subject demographics etc.

\hypertarget{hardware-and-software}{%
\subsection{Hardware and Software}\label{hardware-and-software}}

\hypertarget{motion-capture}{%
\subsubsection{Motion Capture}\label{motion-capture}}

Motion capture data were collected using a 9 (8 Qualisys Miqus M3 marker and 1 Qualisys Miqus Video) camera system connected to a Qualisys Camera Sync Unit.
Marker data were acquired at at a sampling rate of 300 Hz and video data were acquired at a sampling rate of 25 Hz. Qualisys Track Manager software was used to collect and process the data with real-time 3D tracking data output.

\hypertarget{markers}{%
\paragraph{Markers}\label{markers}}

For the experimental set up, one passive marker was placed on each car, and additional passive markers were placed at the start and end of each track, as well as one passive marker on each corner of the surface the tracke was mounted on (see Fig X).
These additional markers provided reference points for 3D orientation of the track and the cars across trials in case of accidental table movement.

\hypertarget{sonification}{%
\subsubsection{Sonification}\label{sonification}}

\hypertarget{hardware}{%
\paragraph{Hardware}\label{hardware}}

Motion capture data were sent via UDP packets over USB networking to a Bela Mini computer running version 0.3.8g running a custom C++ program \footnote{Source, data and analysis are available at https://github.com/zeyus/QTM\_Bela\_Sonification}. The main program loop

\hypertarget{software}{%
\paragraph{Software}\label{software}}

\hypertarget{experimental-design}{%
\subsection{Experimental Design}\label{experimental-design}}

\hypertarget{analysis}{%
\section{Analysis}\label{analysis}}

\hypertarget{results}{%
\section{Results}\label{results}}

\hypertarget{discussion}{%
\section{Discussion}\label{discussion}}

\newpage

\printbibliography[title=References]

\end{document}
